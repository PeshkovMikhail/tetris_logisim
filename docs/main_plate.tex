\documentclass[./main.tex]{subfiles}
\graphicspath{{\subfix{images/}}}
\usepackage{multirow}
\begin{document}
	\section{Main chip}
	Main chip has 3 1-bit inputs for the keyboard, 1 1-bit output for the end of the game, and 6 4-bit outputs to display the score.
	\subsection{IO addresses}
	\begin{center}
		\begin{tabular}{|c | c | c|}
			\hline
			Name & Mode & Address(hex) \\
			\hline
			Control register & w & f0 \\
			\hline
			Write data & w & f1 \\
			\hline
			Read data & r & f2 \\
			\hline
			Figure X coordinate & r/w & f3 \\
			\hline
			Figure Y coordinate & r/w & f4 \\
			\hline
			Score & r/w & f5 \\
			\hline
			Read keyboard & r & f6 \\
			\hline
			Read status & r & f7\\
			\hline
		\end{tabular}
	\end{center}


	Addresses 0xf0 to 0xf7 are reserved for I/O.  Addresses from 0xf8 to 0xff are reserved for the stack. Working with addresses that have the ability to read and write is described in the \hyperref[sec:ioreg]{\textbf{ioreg}} chip.
	\subsection{Control register}
	
	
	This register activates field and figure chips and controls them.
	\begin{center}
		\begin{tabular}{|c | c | p{10cm}|}
			\hline
			Name & Receiver & Purpose \\
			\hline
			csFig & figure & enable figure chip \\ 
			\hline
			csField & field & enable field chip \\
			\hline
			getNew & figure &  return next figure sprite \\
			\hline
			rotate & figure & return next turn sprite \\
			\hline
			check & field & command to check the coordinates of a figure and its position on the field \\
			\hline
			union & field & command to union a figure with a field \\
			\hline
			draw & field & command to draw a figure on the LCD \\
			\hline
			back & figure & return previous turn sprite \\
			\hline
		\end{tabular}
	\end{center}
	\subsection{Keyboard}
	
	
	This part of the chip has 3 inputs, each of which is responsible for the movement of the figure: left, turn, right. When one of the inputs receives a logical one, the value corresponding to the purpose of the input is written to the register in accordance with the table. The register value is cleared after the Cdm-8 reads its value.
	\begin{center}
		\begin{tabular}{|p{3cm}|p{3cm}|p{3cm}|}
			\hline
			\multicolumn{3}{|c|}{Bit mask} \\
			\hline
			left & rotate & right \\
			\hline
			0b00000010 & 0b00000011 & 0b00000001 \\
			\hline
		\end{tabular}
	\end{center}
	\subsection{Game over}
	When it is no longer possible to take the next figure, the program ends. The address of the last command is 0x54. When Cdm-8 reaches it, a logical one appears at the output of gameOver.
\end{document}